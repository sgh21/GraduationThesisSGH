\documentclass{Diploma}
%引入需要的宏包
%\usepackage{amsmath}
%设置封面信息
\SetDepartment{机械工程系}
\SetMajor{机械工程}
\SetTitle{手机BTB插接件位姿识别与机器人柔顺装配技术研究}
\SetAuthor{申广辉}
\SetInstructor{吴丹}{教授} 
% \SetJointInstructor{副指导}{副教授}
\begin{document}
% 摘要
\ChsAbstract 综合论文训练是本科教学的最后一个综合性实践环节,它要求学生根据自己的专业志趣选题,并在教师指导下综合运用已掌握的知识或通过学习新知识,完成一项课题研究或相应的综合训练任务,并独立完成一篇论文,作为学生的“学士学位论文”。
 
本文制定了综合论文的统一格式,避免论文格式上的混乱,同时也减少学生在文档格式编辑上所耗费的时间。另外还总结了少量的论文写作技巧供学生参考。

为了使学生更容易上手\LaTeX ,本文简短的介绍了一些\LaTeX 的常用命令及软件安装等入门知识。

\ChsKeywords{综合论文训练;LaTeX排版;文档格式;写作技巧}

\EngAbstract Diploma project is the final section of comprehensive practice for undergraduate students. The student should determine a project according to whose own professional interesting, and complete the project independently under the instructing of supervisor. An integral thesis paper should be accomplished by student, which is necessary for applying the bachelor's degree.

In this paper, a standard format of thesis in \LaTeX\ system was defined, which can avoid the confusion of different papers from different authors. At the same time, the time consuming of format editing can be saved. Additionally, some of the science writing skills were recommended in this paper.

To make it easier for students to get started with \LaTeX, this paper briefly introduces some common commands and basic knowledge such as software installation for \LaTeX .

\EngKeywords{diploma project; LaTeX editing; document format; writing skills}

% 目录及图表清单
\ListOfContents
% 符号和缩略语说明
\begin{Abbreviations}
\TeX & 一种文字排版系统\\
\LaTeX & \TeX 系统的扩充\\
\LaTeXe & \LaTeX 的中间版本\\
规范 & 《清华大学综合论文训练写作规范(试行)》 \\
环境 & 形如“\char92 begin……\char92 end”的成对指令,指令之间是内容段落\\
命令 & 形如“\char92 TeX”这样的单个指令\\
伪代码 & \LaTeX 文档的源文件,后缀为“tex”\\
\end{Abbreviations}

\StartMainText% 正文开始

\chapter[preface]{引言}
综合论文训练是本科教学的最后一个综合性实践环节,它要求学生根据自己的专业志趣选题,并在教师指导下综合运用已掌握的知识或通过学习新知识,完成一项课题研究或相应的综合训练任务,并独立完成一篇论文,作为学生的“学士学位论文”。综合论文训练的题目,学生即可根据导师(或导师组)的指导选择,也可从大学生研究训练(SRT)项目及各类课外学术科技活动和社会实践中选择,或以其它方式选题。在综合训练过程中,要求学生完成研究问题定义和描述、研究目标确定、问题分析、解决方案及其验证、论文撰写与答辩任务。在教学管理上,综合论文训练设置开题、中期检查和最终答辩三次阶段性检查,指导教师每周对学生工作进展和质量进行一次检查。

作为本科阶段最重要的综合实践训练环节,综合论文训练旨在培养和提高学生解决实际问题的基本能力、创新意识和创新能力,加强学生从事论文(研究)工作的书面和口头表达能力,以及协调组织能力。

为了保障综合论文格式上的一致性,同时也为了尽量减少同学们在格式的手动调整上耗费过多时间,使同学们能够集中精力于课题研究,本文提供了相对规范且便于操作的一套模板及范例供同学们参照使用。

本文所有软件相关内容均以 TeXLive 2024年发行版为例,其中使用了XeTeX 编译系统,并使用了\TeX、\LaTeX 及 \LaTeXe\ 的相关特性。对于其他\LaTeX 版本,相关操作大同小异,如遇操作上的细微差别,请同学们自行解决。

\section{本文的主要内容}
本文的主要内容集中在以下几个方面:

规范化文档格式的定义。参照《清华大学综合论文训练写作规范(试行)》(以下简称“规范”,详细内容参见\ref{txt:require})对格式的要求,对其中涉及的所有格式规定都进行了规范化定义,给出了易于理解和使用的\LaTeX 命令,足够的排版命令可以使同学们在写作时无须顾及格式控制,只需要集中精力撰写文字内容。

充分利用\LaTeX 的自动排版功能。\LaTeX 具备特别强大的排版控制功能,只要合理使用预先定义好排版命令,就可以尽量减少人工干预,既省时省力,又避免了人为造成的格式偏差。

本文的逻辑结构参见\ref{fig:structure}。
\InsertFigure[structure]{}{全文逻辑结构}{structure}%

\section{社会责任分析}
如有可能,可以在论文第1章的最后一节专门阐述本论文研究、开发、设计的技术内容及其工程实施对“社会、健康、安全、法律以及文化”和“环境、社会可持续发展”等制约因素的影响。如:根据论文工作的应用背景,针对性的应用相关知识评价论文工作对这些制约因素的影响,并理解应承担的相应责任。或通过对专业相关领域的技术标准体系、知识产权、产业政策和法律法规的分析,理解不同社会文化对论文相关工程活动的影响。

本小节的内容是非强制性的,如有相关内容,可以展开叙述。如没有相关内容,则可以删除本节。

\chapter{模板的使用方法}
本章主要介绍“综合论文训练模板”的使用方法。首先介绍了一些总体原则,然后对诸如封面等固定格式进行了规定,最后详细介绍了与论文内容相关的常用排版命令。

\section{总的原则}
MS Word还是\TeX ?这实在是一个问题,但又算不上是一个问题。因为虽然Word与\TeX 同为排版软件,但它们的设计理念以及所考虑的用户群是却是大不一样的。Word的设计里念是所见即所得(\textbf{W}hat \textbf{Y}ou \textbf{S}ee is \textbf{W}hat \textbf{Y}ou \textbf{G}et,\textbf{WYSWYG}),也就是说,在Word中,用户可以很容易、很方便地任意修改稿件的样式。并且,用户马上能看到修改后所产生的效果。最重要的一点是,用户在Word中所看到的、即时生成的所有效果,都与最终打印成纸质文件的效果完全相同,这就是WYSWYG的精髓所在。这就好比是Word给了用户一张魔术白纸,用户可以用很直观的方法在上面任意安排所感兴趣的“电子涂鸦”。比真正的涂鸦更为方便的是,用户可以在Word提供的白纸上随意增删和移动这些“电子涂鸦”而不用但心留下任何痕迹。此外,Word还提供了大量方便的工具使得用户能够更好的完成“涂鸦作品”。

很显然,Word的优点在于其强大的WYSWYG,而其缺点,恰恰也在于WYSWYG。由于它会即时地显示排版效果,从而难免令用户分散相当一部分精力在排版的细节问题中。这就使得当用户在编写某些以内容为主的文档时(例如科学论文),不能把全部的精力集中在文件的实际内容中。试想当用户正聚精会神地编写一份报告时,还需要时刻注意当前的排版状态,这就势必会极大的影响写作的效率。另一方面,由于WYSWYG太过方便,用户可以随时随地、随心所欲地更改文档的排版样式,这种随意性很容易导致文档格式的混乱。这对于那些内容冗长,需数日乃至数月数年之功方能完工的大作,WYSWYG的易用性导致的随意性就很有可能使文档格式随着作者的心情变化而变化。例如字体、段落样式、标题样式等元素,在Word中只需要点击几次鼠标就可以作出大量修改。于是经常能看到一篇Word文档中充斥着多种多样的段落样式,各种花哨的字体和标题。当最终成文时,作者往往需要额外耗费大量的精力来统一全文的格式。

与Word相比,\TeX 的优点正在于其严格的格式控制——完全的非WYSWYG,并由此形成了另一种可称为所思即所得的风格(\textbf{W}hat \textbf{Y}ou \textbf{T}hink is \textbf{W}hat \textbf{Y}ou \textbf{G}et, \textbf{WYTWYG})。 在\TeX 中,用户甚至需要对文档进行“编译”后才能看到最终的排版效果,这样的设计自然就会使得用户对随意更改格式存有戒心。尤其对于初级用户,他们可能连更改格式的命令都不能熟知。这是\TeX 的不方便之处,但这样的不方便恰好可以令使用者把所有的精力都用于内容的编撰上,因为\TeX 的排版命令实在是又多又杂又不直观又有兼容性问题。

一言以毕之,从某种程度上来讲Word和\TeX 是互补的,对于科技论文等以内容为主的文档,\TeX 有其独道之处。

本节仅给出一些基本概念和原则,排版命令的详细介绍将在后文中给出。

\subsection{尽量只使用预定义的排版命令}
为了尽可能的避免格式上的混乱,本文强烈建议同学们尽可能避免使用除预定义排版命令之外的其他排版指令。除非预定义的排版命令确实无法满足需求。

例如,“规范”中对一级节标题的格式要求是“四号字;中文黑体、英文Arial;居左书写,行距固定值20磅;段前空24磅,段后空6磅。”在\LaTeX 排版系统里,实现这一格式要求大致需要以下代码:
\begin{verbatim}
\fontsize{14pt}{20pt}%四号字,行距20磅
\setCJKmainfont{SimHei}%中文字体设置为黑体
\setmainfont{Arial}%英文字体设置为Arial
\vspace{24pt}%段前24磅
1.1 一级节标题
\vspace{6pt}%段后6磅
\par %换行
\end{verbatim}
这样的处理方式,显然有两个缺点:一是代码量太大;二是存在不小心写错格式控制量,比如把“24pt”写成“14pt”之类的笔误。避免上述缺点的方式是,使用预定义的排版命令“\char92 section”,此时上述代码段可以被下面的一行简单的代码所取代:
\begin{verbatim}
\section{一级节标题}
\end{verbatim}

本文预定义的排版命令主要有如下几类:
\begin{itemize}
  \item 填充规定信息,如设置指导教师信息的命令“\char92 SetInstructor\{教师名\}\{职称\}”
  \item 文档段落环境,如缩略语段落“\char92 begin\{Abbreviations\}\\
  ……\char92 end\{Abbreviations\}”
  \item 章节标题。如章标题“\char92 chapter[书签名]\{章标题\}”
  \item 插入图片,如“\char92 InsertFigure[书签名]\{宽度\}\{图题\}\{文件名\}”
  \item 表格环境,如“\char92 begin\{table\}[书签名]\{表格名\}\{表格格式\}\{标题行\}\\
  ……\char92 end\{table\}”
  \item 列表环境,如“\char92 begin\{itemize\}……\char92 end\{itemize\}”
\end{itemize}

\subsection{书签和引用}
在论文的文字内容中,常常会需要对图表或公式内容进行描述。例如“如图4.2所示……”,这里的“4.2”是图片的编号。试想,如果“4.2”这几个字是手动输入的,那么一旦图片编号有变动(比如图4.2之前又插入了一幅图片),“4.2”的字样就又需要手动进行改动。如果论文中的图片只有两三张,那么手动处理编号是可以接受的。但若论文中图片数量有几十甚至上百张呢?手工处理就变得难以完成。这时候就需要使用\LaTeX 系统提供的“书签”和“交叉引用”的功能。

所谓的“书签(label)”,可以理解为是给某一段文字取一个“名字”;当需要引用这段文字时,则可以利用“引用(reference)”功能,通过之前所取的“名字”,实现对相应文字的复用。很容易理解,“书签”和“引用”是成对使用的。在\LaTeX 系统里,显式设置书签的命令是“\char92 label”。在本模板中,部分排版命令会自带设置书签的功能。例如一级节标题命令“\char92 section”,可以通过可选参数定义一个书签:“\char92 section[labelname]\{节标题\}”。此时,会自动生成一个名为“txt:labelname”的书签。当需要使用该书签时,可以通过“\char92 ref”命令进行引用,如“\char92 ref\{txt:labelname\}”。


\subsection{自动编号}
仍以“如图4.2所示……”为例,这里的“4.2”是图片的编号。在\LaTeX 系统里,这个编号通常会由一些排版命令或机制自动给出,以避免手动编号引起的混乱。在本模板中,大部分的编号也是由系统自动给出的,包括:章节编号、图表编号、参考文献编号等。这一自动编号过程隐藏在文档类(document class)的代码定义中,同学们使用时无须关心具体地实现过程。通常,\LaTeX 排版文件的第一句代码就是引用“文档类”,如“\char92 documentclass\{Diploma\}”即代表本文使用的是“Diploma”文档类。

\section{固定格式}
本节主要介绍一些固定的格式,所谓的固定格式是指相应设置和内容都已经调整好,不需要使用者再进行改动的相关内容和版面,包括:页面格式设置;页脚内容及格式设置;封面内容及格式设置;授权页内容及格式设置;致谢页格式设置;声明页内容及格式设置;目录内容及格式设置等。这些格式是根据“规范”的要求(详见\ref{txt:require})进行设置的,非必要情况下不需要进行更改。

\subsection{页面、页眉及页脚设置}
页面设置的各项数据如\ref{tab:pageformat},
\begin{table}
  [pageformat]{正文页面设置数据}
  {cccccc}
  {纸张规格 & 上边距 & 下边距 & 左边距 & 右边距 & 装订线 }
  A4, 21.0$\times$29.7cm & 3.0cm & 3.0cm & 3.0cm & 3.0cm & 无 \\
\end{table}%
这些数据在模板里已经设置好,一般情况不需要再进行设置。

页眉是空的,页脚是居中的页码,距页面底边为1.5cm。

\subsection{封面、授权、致谢及声明页}
封面包含以下几种信息:论文题目、专业信息、作者信息、指导教师信息等。相关格式已经预定义好,只需要更改相应内容,不需要再进行格式设置。论文题目小于等于25个字,以导师给的题目为准,简短精练为佳;指导教师信息应包括教师姓名和职称,职称务必与教师确认。

关于学位论文使用授权的说明,需要单独设一页,内容和格式都是固定的,不要改动,打印出来签字即可。授权页在封面之后第一页。

致谢部分,单独一页,请尽量不要罗列与论文内容无关的人员或团体。

声明页,单独一页,格式与内容都是固定的,不要改动,打印出来签字即可。

致谢和声明页放在附录结束之后,研究成果之前。

综合论文训练记载表通常由研究所准备,直接装订到终版综合论文即可。

\subsection{目录}
目录页是自动生成的,相关格式不需要再调整。目录包括1~3级章节,以及参考文献、附录、致谢、声明、研究成果。

\subsection{书脊}
书脊是论文侧面竖排的文字,包括论文题目和作者,便于在书架上查阅。字号根据论文的厚度确定,小四号、四号、小三号均可。书脊需要打印在A3纸张上,因此相关文档及格式见另一文档“综合论文训练书脊”(word格式)。校内的打印店通常会有书脊的模板。

\subsection{插图及附表清单}
插图及表格清单均利用了\LaTeX 的自动清单生成功能,相关格式和清单内容均已设置完毕,不再需要同学们另行处理。

\section{内容相关排版命令}
本节仅对相关排版命令进行简单展示,具体的使用方式请参见\ref{txt:usage}。

\subsection{标题}
推荐使用三级标题结构,标题级别过多容易引起论文结构混乱。各级标题的排版命令分别为“\char92 chapter”、“\char92 section”、“\char92 subsection”。如果确需使用四级标题,排版命令为“\char92 subsubsection”。

上述这几个排版命令的格式是相似的,这里以章标题为例进行说明。章标题命令使用格式为“\char92 chapter[书签]\{标题\}”。其中,“标题”是必填项,内容是本章的标题文字。“书签”是可选项,内容是“书签”的名字,注意这里会跟书签名加上一个前缀“txt:”。例如,下面这个命令将会产生一个名为“txt:preface”的标签:
\begin{verbatim}
\chapter[preface]{引言}
\end{verbatim}
当使用形如“\char92 ref\{txt:preface\}”的引用命令时,将会自动给出“引言”的章编号。

\subsection{图表}
插入图片的基本排版命令为“\char92 InsertFigure”,这个命令会在文档中插入一张图片。
\InsertFigure[singlefig]{}{插入单个图片示例}{sample.png}%
\ref{fig:singlefig}所示为插入单张图片的示例。如果需要在一行中分别展示多张图片,可以使用本文定义的“\char92 begin\{multifigures\}……\char92 end\{multifigures\}”环境,
\begin{multifigures}
\MultiFigure[leftfig]{}{0.49\textwidth}{双图示例之左图}{sample.png}
\MultiFigure[rightfig]{}{0.49\textwidth}{双图示例之右图}{sample.png}
\end{multifigures}%
就像\ref{fig:leftfig}和\ref{fig:rightfig}那样。注意这不是一条简单命令,而是一个“环境”。使用时需要在“环境”中引入相应的排版命令才能获得正确的排版效果。

如果一个图片由多个分图组成,则可以使用子图环境“\char92 begin\{subfigures\}……\char92 end\{subfigures\}”进行展示,
\begin{subfigures}[subfig]{带有分图的大图示例}
\SubFigure[subfiga]{}{0.49\textwidth}{分图甲}{sample.png}
\SubFigure[subfigb]{}{0.49\textwidth}{分图乙}{sample.png}
\end{subfigures}%
具体样例可参看\ref{fig:subfig}。其中分图可以有标题,也可以没有标题,只标出(a)、(b)的序号即可。

插入表格的基本排版命令为“\char92 begin\{table\}……\char92 end\{table\}”。按“规范”的要求,推荐使用三线表格。
\begin{table}
  [sample]{三线表格示例}
  {ccc}
  {表头A & 表头B & 表头C }
  内容A1 & 内容B1 & 内容C1 \\
  内容A2 & 内容B2 & 内容C2 \\
\end{table}%
表格样例如\ref{tab:sample}所示,尽量使表格中的内容居中排版。


\subsection{参考文献}
列出作者直接阅读过、在正文中被引用过、正式或非正式发表的刊物、文献及资料。参考文献的写法应遵循国际上通用的习惯以及我国有关国家标准规定,且应全文统一,不能混用。相关规范建议参考国家标准化管理委员会编撰的《文后参考文献著录规则》(GB/T7714-2015)\cite{gbt7714-2015},图书馆或网上均能查到相关资料。或者也可以参看教务处发布的“参考文献著录格式”文档。

本模板给出了基本的参考文献著录格式规范,主要包含:书籍、书籍中的章节、论文集、论集中的论文、文献汇编、文献汇编中的论文、技术报告、学位论文、标准、期刊论文、新闻等。

\clearpage\subsection{脚注}
脚注主要用于补充说明那些难懂的短语,通常情况下工科学术论文不推荐使用脚注。若确需使用脚注,可以使用排版命令“\char92 footnote”\footnote{这是一条脚注示例}。

\subsection{列表}
本文提供有编号和无编号两种列表。有编号列表为“enumerate”:
\begin{enumerate}
  \item 此段落应用“enumerate”环境即可生成;
  \item 编号默认为阿拉伯数字。
\end{enumerate}
无编号列表环境为“itemize”,范例如下:
\begin{itemize}
  \item 此段落应用“itemize”环境即可生成;
  \item 条目二
\end{itemize}

列表最多可以嵌套3层。但一般不推荐使用嵌套列表,如实在需要使用,建议嵌套2层即可。

\subsection{公式}
简单的公式可使用“equation”环境生成。具体可以参考电子书《一份(不太)简短的 \LaTeXe\ 介绍》\cite{lshort},这份文档可以在TeXLive的安装目录下找到。专业公式的编写是一门不算小的学问,这里就不花费篇幅进行详述了。

\section{\TeX 系统发行版的安装}
众所周知,\TeX 系统最初是由高德纳(Donald E. Knuth)为出版自己编写的《计算机程序设计艺术》一书而专门开发的。高先生认为\TeX 系统应该是完美的,就象圆周率一样,所以\TeX 的版本号不同于一般软件的“1.0”、“2.1”,而是以趋近于圆周率$\pi$为目标。当前的\TeX 系统核心版本为3.141592653,高先生已经封闭了\TeX 的系统核心开发,号称此版本已经是完美版本,改无可改、无需再改。高先生开发\TeX 系统时没有限定版权,所以\TeX 系统可以被自由地使用和分发。当然这也造成了\TeX 系统出现了众多的发行版,令人眼花缭乱。本文推荐安装TeXLive 发行版,安装包可在网站“http://tug.org/texlive”下载。TeXLive是开源软件,可以免费安装使用。进入TeXLive主页(如\ref{fig:texmain}),点击“download”链接,
\begin{multifigures}
\MultiFigure[texmain]{65mm}{0.49\textwidth}{TeXLive 主页示意图}{texlive1.png}
\MultiFigure[texdown]{65mm}{0.49\textwidth}{TeXLive 安装包下载示意图}{texlive2.png}
\end{multifigures}%
进入如\ref{fig:texdown}所示的下载页面。下载windows系统下的安装压缩包文件“install-tl.zip”。

解压缩安装包后运行“install-tl-windows.bat”。在安装界面中,我们可以选择清华校内的安装源,
\InsertFigure[source]{100mm}{清华校内的TeXLive安装源}{texlive3.png}%
如\ref{fig:source}所示。TeXLive系统比较臃肿,如果不想占用太多硬盘空间,可以只安装最基本的四个包:“Essential programs and files”、“TeXworks editor; TL includes only the Windows binary”、“Windows only support programs”及“XeTeX and packages”(如\ref{fig:packages})。
\begin{subfigures}[packages]{最简安装选项}
\SubFigure{74mm}{0.59\textwidth}{}{texlive4.png}
\SubFigure{50mm}{0.39\textwidth}{}{texlive5.png}
\end{subfigures}%
更详细的信息请参见《TeXLive指南》\cite{tlintro}。

注意,模板使用了XeTeX 系统特性,只能使用xelatex命令进行编译。另外,由于模板还使用了BibTeX 系统来管理参考文献,所以完整的编译过程应该是如下四步“xelatex $\rightarrow$  bibtex $\rightarrow$ xelatex $\rightarrow$ xelatex”。

\chapter[usage]{排版命令说明}
\section{\LaTeX 基础}
本节介绍一些特别基础的\LaTeX 排版概念。以下是一个最简单的\LaTeX 文档:
\begin{verbatim}
\documentclass{article}
\begin{document}
Hello world!
\end{document}
\end{verbatim}
这一段“伪代码”包含两个要素。第一行的“\char92 documentclass”命令指明了本文档将使用一个名为“article”的“文档类”,这个“文档类”中含有大量的排版命令和格式定义。通常,一个文档的形貌完全由“文档类”来决定,对于同样的内容“伪代码”,使用不同的“文档类”将得到不同的最终排版效果,而内容“伪代码”几乎不需要被修改。接下来是文档的主要内容,这个主要内容一定要用“\char92 begin\{document\}”和“\char92 end\{document\}”两个命令“包”起来。

在“\char92 documentclass”和“\char92 begin\{document\}”之间,可以插入“\char92 usepackage”命令或其他的一些设置文档全局信息的命令。“\char92 usepackage”命令是引入特定的“宏包”,并实现特定的功能。

在Word软件中,通过输入“回车”可以产生一个新的“段落”;而“换行”则可以通过输入“SHIFT+回车”实现。在\LaTeX 中,产生一个新的“段落”一般需要在“伪代码”中空出一行,例如以下段落:
\begin{verbatim}
第一段

新的段落
\end{verbatim}
而换行则需要显式地使用换行命令“\char92 newline”。

在中文\LaTeX 环境中,“伪代码”里中文之间的空格会被忽略,这一点与Word软件有很大的不同。对于英文间的空格,\LaTeX 会忽略其数量,即“1\begingroup \setmainfont{Latin Modern Roman}\textvisiblespace \endgroup 2”和“1\begingroup \setmainfont{Latin Modern Roman}\textvisiblespace \textvisiblespace \endgroup 2”的排版效果是一样的,并不会因为空格更多就会在文字间产生更长的空白。当需要指明空格的数量时,可以使用所谓的空格命令“\char92 \begingroup \setmainfont{Latin Modern Roman}\textvisiblespace \endgroup”。这个“空格命令”能够显式地产生一个空白,并且这个空白的长度是可以叠加的。

关于字体,\LaTeX 的控制命令相对比较复杂,不能像Word软件那样点一下字体菜单就能够“轻松”地更换一种字体。这里我们不推荐用复杂的\LaTeX 命令去“手动控制”字体。如果文档中需要使用一些常见的字体,例如“黑体”和“斜体”,建议使用标准的字体控制命令来实现。临时切换成“黑体”可以用命令“\char92 textbf\{文字内容\}”来实现;相应地,临时切换成“斜体”的命令是“\char92 textit”。还有一常用的字体控制命令,就是实现“上标”的功能,这个命令是“\char92 textsuperscript”,例如“\char92 textsuperscript\{这是上标\}”的实际效果是\textsuperscript{这是上标}。

在\LaTeX 系统中,常用的长度单位有mm、in(英寸,25.4mm)、pt(磅,1/72.27in)、em(当前字体条件下大写字母M的宽度)、ex(当前字体条件下小写字母x的高度)。一般情况下,使用长度时应显式地给出单位。

在\LaTeX 的“伪代码”中,存在一些特殊字符,例如“\%”、“\$”、“\{”、“\}”、“\char92 ”等。若想在最终的排版结果显示这些特殊字符,则需要使用转义字符“\char92 ”,如“\$”是通过排版命令“\verb|\$|”得到的。转义字符“\char92 ”本身不能通过“\char92 \char92 ”得到,因为“\char92 \char92 ”是换行命令,应使用“\char92 char92”得到“\char92”。

\section{固定样式}
在正文开始之前,即“\char92 begin\{document\}”命令前面,必须设置好论文的封面信息,包括:
\begin{itemize}
  \item 设置系名:\char92 SetDepartment\{系名\}
  \item 设置专业名:\char92 SetMajor\{专业名\}
  \item 设置论文题目:\char92 SetTitle\{论文题目\}
  \item 设置作者名:\char92 SetAuthor\{作者名\}
  \item 设置指导教师:\char92 SetInstructor\{指导教师名\}\{指导教师职称\}
  \item 设置副指导教师(如有):\char92 SetJointInstructor\{副指导教师名\}\{副指导教师职称\}
\end{itemize}
以上命令用法比较简单,这里就不再赘述。如果设置了副指导教师,则封面会出现副指导教师的相关信息。综合论文训练一般不设置副指导教师。

除了封面和授权页之外,还有以下部分也属于固定样式:目录、插图清单、附表清单、声明页。其中,目录、插图清单、附表清单均由“\char92 ListOfContents”命令自动生成;声明页由“\char92 Statement”命令生成。

固定样式命令应出现的位置详见附录\ref{txt:tplstruct}。

\section{排版命令}
\subsection{标题相关}
与标题相关的排版命令如\ref{tab:commands}所示。
\begin{table}[commands]
{标题相关的排版命令列表}{lcc}
{排版命令 & 功能 & 说明}
\char92 ChsAbstract & 中文摘要页 & \\
\char92 EngAbstract & 英文摘要页 & \\
\char92 chapter[label]\{title\} & 章 & label是可选参数 \\
\char92 section[label]\{title\} & 节 & 同上 \\
\char92 subsection[label]\{title\} & 小节 & 同上 \\
\char92 subsubsection[label]\{title\} & 小小节 & 同上,不推荐使用 \\
\char92 Acknowledgments & 致谢 & \\
\char92 Achievements & 成果 & 在学期间研究成果 \\
\end{table}%
下面会详述这些排版命令的使用方法。

在中/英文摘要内容开始之前,应分别加上\char92 ChsAbstract 和\char92 EngAbstract。

在章节内容前应加上章节命令。以\char92 chapter为例,以下代码将在文档中产生名为“引言”的章名,并在自动生成相应的目录。
\begin{verbatim}
\chapter[preface]{引言}
\end{verbatim}
其中的“preface”是标签名,这是可选项,如果设置了此选项,则会产生一个名为“txt:preface”的标签。注意,小小节属于四级标题,不推荐使用,一般论文安排三级标题即可,四级标题有些过于琐碎。

论文附录之后的三项内容分别是致谢、声明和成果。其中声明页内容是固定的,不能由作者自行编写。致谢、成果页与中/英文接要一样,需要作者在标题之后写出具体内容。

\subsection{符号和缩略语列表}
如需列出符号和缩略语,请按如下范例使用“Abbreviations”环境:
\begin{verbatim}
\begin{Abbreviations}
短语 & 短语的简要说明 \\
\end{Abbreviations}
\end{verbatim}
容易看出,每行缩略语由两段组成,分别为短语及短语的说明,两者用“\&”符号隔开。每行的结尾必须以双斜杠“\char92 \char92”结束。短语尽量不要超过14个英文字符;短语的简要说明尽量不要超过60个英文字符。

\subsection{插入图片}
插入图表相关的排版命令如\ref{tab:figtab}所示。
\begin{table}[figtab]
{图表相关的排版命令列表}{lcc}
{排版命令 & 功能 & 说明}
\char92 InsertFigure & 插入单张图片 & 一张图片占一行 \\
multifigures环境 & 定义多图环境 & 可在一行中放置多张图片 \\
\char92 MultiFigure & 插入多图 &  仅可用于“多图环境” \\
subfigures环境 & 定义子图环境 &  以子图形式在一行中放置多张图 \\
\char92 SubFigure & 插入子图 & 仅可用于“子图环境” \\
table环境 & 定义表格环境 & \LaTeX 标准语法的表格 \\
\end{table}%
下面详述它们的使用方法。

插入单张图命令的完整格式是“\char92 InsertFigure[label] \{图片宽度\} \{图题\} \{文件名\}”,其中各参数的说明如下:
\begin{itemize}
  \item label,标签。可选项。若设置此项,则会产生一个名为“fig:label”的标签。
  \item 图片宽度。宽度参数可以为空,即“\{\}”,此时插图宽度由图片本身决定。
  \item 图题,图片的标题。这是必填的参数。
  \item 文件名,含后缀。具体支持的格式需参考\TeX 编译系统。注意本模板设置图片的\textbf{默认目录为“figure”},所以需要将图片存入“figure”目录。
\end{itemize}
如果设置了标签项,则“\char92 InsertFigure”命令会自动为图题编号,并加入插图清单。注意,若需要产生形如“图x.x”这样的图题编号,则必须设置“[lable]”标签项,否则图表标题中只有标题文字,没有编号,也不进入插图清单。

定义多图环境命令的完整格式是\\
“\char92 begin\{multifigures\} …… \char92 end\{multifigures\}”。\\
在多图环境中,我们才可以使用“\char92 MultiFigure”命令,来实现在一行中插入多张图片。“\char92 MultiFigure”命令的完整格式是“\char92 MultiFigure [label] \{图片宽度\} \{栏宽度\} \{图题\} \{文件名\}”,其中各参数的说明如下:
\begin{itemize}
  \item label,标签。可选项。若设置此项,则会产生一个名为“fig:label”的标签。
  \item 图片宽度。该参数可以为空,此时图片宽度由其本身决定。
  \item 栏宽度。必填参数,不能为空。参数需要带长度单位。在多图环境下,一个文本行会被分成若干“栏”,并在每个“栏”中插入图片,这个“栏宽度”参数将指明“栏”的宽度。注意,“栏”宽度只影响排版位置以及图题的占用宽度,不影响图片的实际大小。通常,栏宽可以将页面宽度用作单位。例如,“0.5\char92 textwidth”代表栏宽为半个页面宽度。
  \item 图题,图片的标题。这是必填的参数。
  \item 图片文件名。
\end{itemize}
与“\char92 InsertFigure”命令相同,在设置了标签项的情况下,“\char92 MultiFigure”命令也会自动为图题编号,并加入插图清单中。

定义子图环境命令的完整格式是:\\
“\char92 begin\{subfigures\} [label] \{图题\} …… \char92 end\{subfigures\}”,\\
其中各参数的说明如下:
\begin{itemize}
  \item label,标签。可选项。若设置此项,则会产生一个名为“fig:label”的标签。
  \item 图题,图片的标题。这是必填的参数。
\end{itemize}
在子图环境中,我们才可以使用“\char92 SubFigure”命令,来实现在一行中插入多张子图片。“\char92 SubFigure”命令的完整格式是“\char92 SubFigure [label] \{图片宽度\} \{栏宽度\} \{子图题\} \{文件名\}”,其中各参数的说明如下:
\begin{itemize}
  \item label,标签。可选项。若设置此项,则会产生一个名为“fig:label”的标签。另外,如果设置了标签,则会自动为子图的标题加上形如“(a)”、“(b)”的子编号。若未设置标签,则没有子编号。
  \item 图片宽度。该参数可以为空,此时图片宽度由其本身决定。
  \item 栏宽度。必填参数,不能为空。参数需要带长度单位。在子图环境下,一行被分成若干“栏”,并在每个“栏”中插入图片,“栏宽度”参数指明“栏”的宽度。栏宽只影响排版位置以及图题的占用宽度,不影响图片大小。
  \item 子图题,子图片的标题。可以为空。注意,当标签和子图题均为空时,子图片将会没有任何标题信息。
  \item 图片文件名。
\end{itemize}
需要特别说明的是,子图环境本身会有一个总的图题,当我们为子图环境设置了标签项时,模板会自动为这个总图题编号,并加入插图清单中。注意,只有在子图环境设置了标签项时,才能为其中的子图单独设置标签;即使子图设置了标签,也不会被加入到插图清单中(插图清单中只列出总图题),但可以使用“\char92 ref”命令引用相应的标签。当引用子图环境的总图题时,将不包含子图的编号“(a)”、“(b)”,例如“\ref{fig:subfig}”;当引用子图题时,则会包含子图的编号,例如“\ref{fig:subfiga}”。

\subsection{插入表格}
定义表格环境命令的完整格式是:\\
“\char92 begin\{table\} [label]  \{表题\} \{格式控制\} \{标题行\} …… \char92 end\{table\}”\\
其中各参数的说明如下:
\begin{itemize}
  \item label,标签。可选项。若设置此项,则会产生一个名为“tab:label”的标签。
  \item 表题,表格的标题。这是必填的参数。
  \item 格式控制,符合\LaTeX 标准语法的格式控制参数序列。常用的有三种控制字,分别是“c”、“l”及“r”,分别代表“居中”、“左对齐”及“右对齐”。表格的列数由控制字的数量决定。例如“ccc”代表一个3列的表格,每一列都是居中对齐模式。更详细的使用说明请参见《一份(不太)简短的 \LaTeXe\ 介绍》\cite{lshort}。
  \item 标题行,表格的首行。首行一般是每一列的内容的标题。如果标题需要切换成黑体字,可使用字体命令“\char92 textbf”。
\end{itemize}
在设置了标签项的情况下,“table”环境会自动为表题编号,并加入到附表清单中。按照“规范”的要求,本文中的表格均为简单的“三线表”,即表格起止用粗线标出、首行用细线隔出。

在\LaTeX 中,表格中的各列的内容用“\&”符号隔开;每行内容以双斜杠“\char92 \char92”结束。表格的使用限制比较多,涉及的排版命令比较繁杂,尤其是编制复杂表格时,需要用到一些特殊的排版技巧。推荐只使用简单表格,如确需用到复杂表格,仍请参见《一份(不太)简短的 \LaTeXe\ 介绍》\cite{lshort}。

默认的表格只支持单行文字,若需要填写多行内容,请使用“\char92 parbox”命令。“\char92 parbox”主要有两个参数,第一个参数是栏宽,第二个参数是表格内容,内容中可以换行。例如:
\begin{verbatim}
\parbox{30mm}{……}
\end{verbatim}
以上代码会生成一个宽度为30mm的“box”,这个box中的内容可以是多行的。

另外,需要特别说明的是,图片及表格都属于\LaTeX 中的浮动体(float environment)。浮动体的位置不是精确的,\LaTeX 系统不会强制它们停留在它们最初出现的上下文位置。这对于文档的合理布局非常有用,尤其是在浮动体较多时,\LaTeX 系统会灵活地处理它们,以免页面出现大块留白等不那么美观的现象。当然也会有一些缺点,例如浮动体可能先于其介绍文字出现,或是在介绍文字出现之后较远的位置才出现,此时就需要作者使用“\char92 clearpage”等命令进行精调。

\subsection{插入公式}
编排公式是\TeX 系统的强项,通常情况下,使用“equation”环境即可完成大部分的排版需求,如\eqref{equ:sample}所示。
\begin{equation}\label{equ:sample}
(a+b)^2=a^2+2ab+b^2
\end{equation}
更复杂的公式排版技巧,请参阅经典文献《一份(不太)简短的 \LaTeXe\ 介绍》\cite{lshort}。

在引用公式编号时,请使用专门的命令“\char92 eqref”,这个命令是符合“规范”的,它会自动按“式(x-x)”的格式引用公式编号。

\subsection{参考文献}
参考文献的管理和使用,需要借助BibTeX工具。引入BibTeX工具后,一套完整的\LaTeX 文档编译流程应当是“xelatex project” $\rightarrow$ “bibtex project” $\rightarrow$ “xelatex project” $\rightarrow$ “xelatex project”。一共四步,第1步是粗编译生成引用记录,第2步是用BibTeX工具生成文献列表,第3步是粗编译生成引用标签,第4步才是输出最终的正确结果。

在生成具体的文献列表时,BibTeX需要依赖专门的文献格式定义文件“*.bst”。通常需要在文档中使用“\char92 bibliographystyle”命令显式地对“*.bst”文件进行引用。在本文提供的模板中,已经在模板定义中引用了“mybst.bst”文件,不需要使用者再进行重复操作。

本文提供的模板中包含“mybst.bst”文件,此文件是根据国标文件《文后参考文献著录规则》\cite{gbt7714-2015}编写的,符合“规范”的要求。使用者只须根据“mybst.bst”的要求准备好相应的“*.bib”文献库文件,并在需要引用参考文献的正文段落中,使用“\char92 cite”命令进行引用即可。

引用参考文献的命令的完整格式是“\char92 cite\{文献1,文献2,……\}”。其中不同文献用英文的逗号隔开。“文献1”、“文献2”指的是文献的代码,这是在“*.bib”文献库文件中定义的。“\char92 cite”命令会自动对参考文献进行排序,因此在同一个cite命令中的文献的先后顺序是不重要的。

“*.bib”文献库文件的基本格式为:
\begin{verbatim}
@entry{code,
   info = {content},
}
\end{verbatim}
其中各项参数的说明如下:
\begin{itemize}
  \item entry,文献类型。常用的是这四种类型:“article”期刊论文、“inproceedings”会议论文、“book”书籍、“thesis”学位论文。不常用的类型还有这些:“report”报告、“standard”标准、“incollection”汇编中的论文、“proceedings”会议论文集、“collection”论文汇编集、“inbook”书籍中的章节、“periodical”连续出版物、“news”新闻。
  \item code,文献代码。给文献取一个具备唯一性的代码,以便使用“\char92 cite”命令进行引用。
  \item info,文献信息名。所谓文献信息,即与文献相关的数据内容,例如“author”作者、“title”标题等。对于不同的类型的文献,需要提供的文献信息是不同的。下文将会有更为详细的说明。
  \item content,文献信息的实际内容。
\end{itemize}
不同的文献类型包含不同的文献信息,以常用的几种文献类型为例:
\begin{itemize}
  \item “book”书籍类文献一般包含“author”作者名、“title”书名、“address”出版地、“publisher”出版者、“year”出版年等文献信息;
  \item “article”期刊论文类文献一般包含“author”作者名、“title”论文名、“journal”期刊名、“year”出版年、“volume”卷号、“number”期号、“pages”页码范围等文献信息;
  \item “thesis”学位论文一般包含“author”学位获得者名、“title”论文名、“address”取得学位论文的城市、“publisher”取得学位论文的大学、“year”取得学位的年份等文献信息。
\end{itemize}
“language”信息指明了文献的语言,如果设置成“cn”,则文献为中文,否则系统均默认为英文。通常情况下,作者名为中文的文献,请设置语言为中文。附录中的节\ref{txt:bibref}列出了不同的文献类型的数据信息列表及详细说明信息。

在需要列出参考文献的位置,应插入命令“\char92 bibliography\{*.bib\}”。除了正文中的参考文献之外,在“附录A”中也可能会出现参考文献。这部分参考文献不能使用“bibliography”命令自动生成,应使用“reflist”环境手动生成,伪代码如下:
\begin{verbatim}
\begin{reflist}
  \item 参考文献
  ……
\end{reflist}
\end{verbatim}

\subsection{内容分隔相关}
在正文开始前,需要显式地使用命令“\char92 StartMainText”;相应地,在附录开始前,需要显式地使用命令“\char92 StartAppendix”。这两个命令主要用于设置一些格式变量,以标示出正文内容和附录内容开始的位置。注意,此类命令只出现一次,不要重复使用。

\chapter{一些写作技巧}
建议先定出总的提纲。提纲是论文总体构想的体现,编写提纲是一个清理思路、安排材料,组织结构的过程。提纲对论文的整体把控十分重要:利于论文的谋篇布局;利于论文的整体进程;利于论文的写作安排。拟定提纲的要点:反映自己的思路,要做到条理清楚;突出论文每一章节的要点。

对于一般科技论文,建议分为以下三大组成部分:引言,即论文的背景、研究目标和内容;理论/建模与仿真/试验设计/数据分析,即论文的主要研究内容;总结与展望,对全文进行总结和升华。

提纲确定之后,即可着手撰写初稿。撰写初稿时主要注重论文的逻辑性、完整性和章节的联系,注意几个关键点:一旦确定思路,尽快完成、一气呵成;不必过多推敲细节和斟酌词句;写作中可适当调整提纲。

初稿完成后应该进行多个轮次的修改。综合论文的撰写是一个循序渐进的过程,好的论文都是修改出来的,修改是贯穿写作全过程的一项工作。撰写论文是一项严肃的工作,来不得半点浮躁和轻率。这就要求我们具有认真、严谨的治学态度。有了这种态度,我们能对论文的观点反复提炼、资料反复印证、结构反复思考、文字反复斟酌,真正做到一丝不苟。修改论文的几种主要方法:
\begin{itemize}
  \item 请指导老师评阅提出修改建议:不要害怕麻烦导师;
  \item 同学/同行修改:当局者迷,多请教旁观者是有益的;
  \item 冷修改:即写完相对完整的一部分内容后,先不去想它,过几天再返回来修改。这样做可以避免思维定势,能够发现一些被忽略的细节;
  \item 热修改:趁热打铁,即写完一段后马上进行修改,这样做可以使得核心内容的细节更丰富、更饱满;
  \item 读改法:一边朗读,一边思考和修改。如果行文有不通畅之处,朗读时就会觉得别扭,甚至读不下去。因此,读改法是保障行文流畅的不二法门。
\end{itemize}

撰写论文时要注意论文的整体性,注重各章节间的联系和过渡,论文始终围绕一个核心进行系统性的论证与分析:每个章节有要点,主要笔墨应围绕该要点展开;章节之间要有合理的联系,在章节的连接处安排承上启下的段落。

更多的学位论文撰写经验推荐大家关注一个叫“学位与写作”的微言公众号。

\chapter{总论与展望}
本文制定了本科生综合论文训练的统一格式,可有效避免论文格式上的混乱,同时也减少学生在文档格式编辑上所耗费的时间。

在此基础上,也总结了极少量的论文写作技巧供读者参考。

未来,随着更多优秀综合论文的出现,本文会持续的汲取优秀论文中的精华,及时总结那些行之有效的技巧和方法,争取为保障综合论文训练的写作水准提供实用的基础工具。

\section{更新日志}
\noindent\textbf{2025-04-06}

修订了浮动体位置控制选项,按经验设置成“!htb”。

抑制引用文献命令“\char92 cite”在正文前的目录或清单中展开。简言之,当我们在图题中使用“\char92 cite”命令,它在插图清单中会自动失效。

增加了专门的公式引用命令“\char92 eqref”。

修订了插图清单和附表清单命令,当论文中不出现插图时,则不会显示插图清单页;当不出现表格时,则不会显示附表清单。

仔细调整了各种垂直间距,包括图表、图题、表题、章节标题等,基本上能够满足“规范”的要求。但也给出了较大的调整空间,当排版出现困难时,允许\LaTeX 系统在较大范围内调整这些间距。所以,若在部分段落中出现了较大或较小的间距,一般不需要特别去调整,这已经是算法能给出的最优选择。

\noindent\textbf{2025-04-04}

修订了BibTeX 格式定义文件“mybst.bst”,调整文献类型名和数据字段名,使其更符合一般规范。

调整了插入图片命令中的宽度参数,不再限制该参数的单位为“mm”。涉及“InsertFigure”、“MultiFigure”和“SubFigure”三个插图命令。

\noindent\textbf{2025-03-27}

根据《清华大学综合论文训练写作规范(试行)》(2024),发布首个版本。

\bibliography{references}%声明并放置参考文献

\StartAppendix% 附录开始
\chapter{外文资料的调研报告(或书面翻译)}
\begin{center}
调研阅读报告题目(或书面翻译题目)
\end{center}

写出至少 5000 外文印刷字符的调研阅读报告或者书面翻译 1-2 篇(不少于2 万外文印刷符)。

这是附录中的插图示例,
\InsertFigure[appdxfig]{}{附录中的插图示例}{sample.png}%
附录中的图片编码前冠以附录的序号,例如“\ref{fig:appdxfig}”。表格及公式亦如是,这里不再赘述。

\begin{center}
参考文献(或书面翻译对应的原文索引)
\end{center}
\begin{reflist}
  \item 调研报告中的参考文献,请自行编号。
\end{reflist}

\chapter[require]{清华大学综合论文训练写作规范(试行)}
在综合论文训练阶段,本科生须在指导教师的指导下针对某一课题进行探讨、分析和研究,并完成一篇论文。学生应遵照本规范的具体要求进行撰写。原则上,本科生(含国外来华留学本科生)非外语专业论文统一要求用汉语书写。

\section{论文组成部分及顺序}
论文应包含以下部分,顺序如下:
\begin{itemize}
  \item 封面
  \item 关于论文使用授权的说明
  \item 摘要
  \item Abstract
  \item 目录
  \item 插图和附表清单(如有)
  \item 符号和缩略语说明(如有)
  \item 正文:第1章(引言或绪论),第2章,……,结论
  \item 参考文献
  \item 附录
  \item 致谢
  \item 声明
  \item 在学期间参加课题的研究成果(如有)
  \item 综合论文训练记录表
\end{itemize}
以上各项均独立成为一部分,每部分从新的一页开始。

\section{论文格式要求}

\noindent\textbf{(一)封面}

\textbf{题目:}论文题目严格控制在25个汉字(符)以内。字体采用黑体一号字,居中书写。一行写不下时可分两行写,并采用1.25倍行距,断行应合理,应保持术语和词语连续。

\textbf{系别:}院(系)名称的全称。

\textbf{专业:}以本年级《学生手册》中的清华大学本科专业设置为准。

\textbf{姓名:}填写论文作者姓名。

\textbf{指导教师:}填写指导教师姓名,后衬指导教师专业技术职务,如“教授”、“研究员”等,副指导教师、联合指导教师与此相同。

系别、专业、姓名及指导教师信息部分使用仿宋三号字。若不超过4个汉字,作者姓名和指导教师姓名应等宽,各自应保持均匀间隔。

\textbf{论文成文打印日期:}填写论文成文打印的日期,用宋体三号字,不用阿拉伯数字。

\noindent\textbf{(二)关于论文使用授权的说明}

单设一页,排在封面后。内容为“本人完全了解清华大学有关保留、使用综合论文训练论文的规定,即:学校有权保留论文的复印件,允许论文被查阅和借阅;学校可以公布论文的全部或部分内容,可以采用影印、缩印或其他复制手段保存论文。”

此部分内容可以直接下载《清华大学综合论文训练写作规范(试行)》附件的WORD 文档,相应地复制到论文中即可,在提交论文时作者和指导教师都必须签署姓名。

\noindent\textbf{(三)摘要}

\noindent 1. 中文摘要

此部分单设一页。标题为“摘要”,用黑体三号字,居中书写,段前空24磅,段后空18磅,单倍行距。内容部分采用宋体小四号字,两端对齐,行距用固定值20 磅,段前后0磅。

论文摘要中不要出现图片、图表、表格或其他插图材料。

关键词是为了文献标引工作、用以表示全文主要内容信息的单词或术语。关键词3~5个,另起一行,排在摘要的左下方,每个关键词之间用分号间隔。

\noindent 2. 英文摘要

中文摘要页后为英文摘要,单设一页。标题为“Abstract”,用Arial体三号字,居中书写,段前空24磅,段后空18磅,单倍行距。内容采用Times New Roman体小四号,两端对齐,行距用固定值20磅,段前后0磅,标点符号用英文标点符号。“Keywords”与中文摘要部分的关键词对应,每个关键词之间用分号间隔。

论文摘要的中文版与英文版文字内容要对应。

\noindent\textbf{(四)目录}

目录是论文各组成部分章、节序号和标题行以及页码按顺序的排列,列至二级节标题(例如1.2.5)即可。目录内容从正文部分开始,直至论文结束。

目录的标题用黑体三号字,居中书写,单倍行距,段前空24磅,段后空18磅。目录中的章标题行居左书写,一级节标题行缩进1个汉字符,二级节标题行缩进2个汉字符。目录中的章标题采用小四号字,中文采用黑体,英文和数字采用Arial体。其他内容采用宋体小四号字,行距为固定值20磅,段前、段后均为0磅,英文和数字用Times New Roman体。章标题和节标题要简洁,尽可能保持在一行内,若确有必要超过一行,采用悬挂对齐的方式。

目录宜在文档编辑软件中自动生成,并根据上述要求调整格式。

\noindent\textbf{(五)插图和附表清单}

论文中插图和附表较多时,应分别列出“插图清单”和“附表清单”。插图清单在前,应列出图序、图题和页码。附表清单在后,应列出表序、表题和页码。

插图较多而附表较少、或者插图较少而附表较多、或者二者均较少时,可将插图和附表合在一起列出“插图和附表清单”,插图在前、附表在后。

插图和附表清单另起一页,置于目录之后。

章标题“插图清单”“附表清单”或“插图和附表清单”使用黑体三号字,居中,段前空24磅,段后空18磅,单倍行距。内容部分中文采用宋体小四号字,英文和数字采用Times New Roman体小四号,行距为固定值20磅,段前、段后均为0磅。图表标题应简洁,尽可能保持在一行内,若确有必要超过一行,采用悬挂对齐的方式。图中的分图无需在图表清单中体现。

插图与附表清单宜在文档编辑软件中自动生成,并根据上述要求调整格式。

\noindent\textbf{(六)符号和缩略语说明}

如果论文中使用了大量的物理量符号、标志、缩略词、专门计量单位、自定义名词和术语等,应编写“符号和缩略语说明 ”。如果符号和缩略词使用数量不多,可以不设专门的“符号和缩略语说明”,而在论文中出现时随即加以说明。

章标题“符号和缩略语说明”使用黑体三号字,居中书写,单倍行距,段前空24 磅,段后空18磅。内容部分采用宋体小四号字,行距为固定值20磅,段前、段后均为0磅。英文和数字用Times New Roman体。

\noindent\textbf{(七)正文}

\noindent 1. 一般要求

此部分是论文的主体,应从另页右页开始,每一章应另起页。主体部分一般从引言(绪论)开始,以结论结束,分章节论述,层次分明、逻辑性强。

\noindent 2. 标题格式

\textbf{各章标题,例如:“第1章 引言”}

章序号采用阿拉伯数字,章序号与标题名之间空一个汉字符。采用三号字,居中书写,中文采用黑体,英文和数字采用Arial体,单倍行距,段前空24磅,段后空 18 磅。论文的摘要、目录、插图和附表清单、符号和缩略语说明、参考文献、附录、致谢、声明、综合论文训练记录表等部分的标题与章标题属于同一等级,也使用上述格式;英文摘要部分的标题“Abstract”采用Arial体三号字。

\textbf{一级节标题,例如:“2.1 实验装置与实验方法”}

节标题序号与标题名之间空一个汉字符(下同)。采用四号(14pt)字居左书写,中文采用黑体,英文和数字采用Arial体,行距为固定值20磅,段前空24磅,段后空6磅。

\textbf{二级节标题,例如:“2.1.1 实验装置”}

采用13pt字居左书写,中文采用黑体,英文和数字采用Arial体,行距为固定值20磅,段前空12磅,段后空6磅。

\textbf{三级节标题,例如:“2.1.2.1 归纳法”}

采用小四号(12pt)字居左书写,中文采用黑体,英文和数字采用 Arial 体,行距为固定值20磅,段前空12磅,段后空6磅。一般情况下不建议使用三级节标题。

\noindent 3. 论文段落的文字部分

采用小四号(12pt)字,汉字用宋体,英文用Times New Roman体,两端对齐书写,段落首行左缩进2个汉字符。行距为固定值20磅(段落中有数学表达式时,可根据表达需要设置该段的行距),段前空0磅,段后空0磅。
上述论文段落文字格式亦适用于正文后附录、致谢等部分的段落文字。

\noindent 4. 注释脚注

当论文中的字、词或短语,需要进一步加以说明,而又没有具体的文献来源时,用注释,采用文中编号加“脚注”的方式。在正文中需要注释的句子结尾处用%
\begingroup\setmainfont{SimSun}\textsuperscript{①②③}\endgroup
……样式的数字编排序号,以“上标”字体标示在需要注释的句子末尾。在当页下部书写脚注内容。

脚注内容采用小五号字,中文用宋体,英文和数字用Times New Roman体,两端对齐格式,段前后均空0磅,单倍行距,悬挂缩进1.5字符。脚注的序号按页编排,不同页的脚注序号无须连续。

论文中应注意区分各种字符的正斜体、黑白体、大小写、上下角标、上下偏差等。

\clearpage\noindent 5. 字体、字型、字号及段落格式要求表
\begin{table}[format1]
{字体、字型、字号及段落格式要求表}{p{3em}p{4.5em}p{5em}p{6.5em}p{14.5em}}
{& \textbf{文字举例} & \textbf{中文字体、\newline 字号要求} & \textbf{英文及数字字体、字号要求} & \textbf{其他格式要求}}
\textbf{章标题} & 第1章 & 黑体三号字 & Arial三号 & 居中书写,单倍行距,段前空24磅,段后空18磅 \\
\textbf{一级节标题} & 4.1 实验方法 & 黑体四号字 & Arial体14pt & 居左书写,行距为固定值20磅,段前空24磅,段后空6磅 \\
\textbf{二级节标题} & 3.2.2 实验 装置 & 黑体13pt & Arial体13pt & 居左书写,行距为固定值20磅,段前空12磅,段后空6磅 \\
\textbf{三级节标题} & 5.3.3.2 原\newline 材料 & 黑体小四号字 & Arial体12pt & 居左书写,行距为固定值20磅,段前空12磅,段后空6磅。 \\
\textbf{正文} & 实验预期效果 & 宋体小四字 & Times New Roman 12pt & 两端对齐书写,段落首行左缩进2个汉字符。行距为固定值20磅(段落中有数学表达式时,可根据表达需要设置该段的行距),段前空0磅,段后空0磅 \\
\textbf{图题} & 图1.1 达\newline 芬奇系列医疗手术机器人 & 宋体五号字 & Times New Roman 11pt & 居中书写,段前空6磅,段后空12磅,单倍行距,图序与图题文字之间空一个汉字符宽度 \\
\textbf{表题} & 表2.13飞\newline 行时间质谱装置 & 宋体五号字 & Times New Roman 11pt & 居中书写,段前空12磅,段后空6磅,行距为单倍行距,表序与表题文字之间空一个汉字符宽度。 \\
\textbf{参考文献} & [1] 作者.\newline 文题…… & 宋体五号字 & Times New Roman 11pt & 行距采用固定值16磅,段前空3 磅,段后空0磅。采用悬挂格式,悬挂缩进2个汉字符或1个厘米。 \\
\textbf{脚注} & 源于…… & 宋体小五号字 & Times New Roman 小五号 & 两端对齐格式,悬挂缩进1.5字符,段前后均空0pt,单倍行距。\\
\end{table}

\noindent\textbf{(八)量和单位}

严格执行国家标准GB 3100—1993、GB/T 3101—1993和GB/T 3102—1993有关量和单位的规定。单位名称的书写,可以采用国际通用符号,也可以用中文名称,但全文应统一,不得两种混用。

\noindent\textbf{(九)有关图、表和表达式}

图、表和表达式一律采用阿拉伯数字编号,并按章编号,前一位数字为章的序号,后一位数字为本章内图、表或表达式的顺序号。两数字间用小数点“.”或半角横线“-”连接。例如“图2.1”或“图2-1”,“表3.1”或“表3-1”等。表达式在文字叙述中采用“式(3-1)”或“式(3.1)”形式,在编号中用“(3-1)”或“(3.1)”形式。若图或表中有附注,采用英文小写字母顺序编号,附注写在图或表的下方。

附录中图、表、表达式的编号,应与正文中的编号区分开,即在阿拉伯数码前冠以附录的编号,如附录A中的图和表,表示为“图A.1”,“表A.2”等。

\noindent 1. 图

图包括曲线图、构造图、示意图、框图、流程图、记录图、地图、照片等。图应具有“自明性”,即只看图、图题和图例,不阅读正文,就可理解图意。图要精选,切忌与表及文字表述重复。图中的术语、符号、单位等应与正文表述中所用一致。

图应有编号和图题,例如:“图 2.1 发展中国家经济增长速度的比较(1960-2000)”。 图 2.1 是编号,代表“第2章第1个图”,以此类推,“发展中国家经济增长速度的比较(1960-2000)”是图题。图的编号与图题置于图下方,采用11pt字居中书写,汉字用宋体,英文和数字用Times New Roman体,段前空6磅,段后空12磅,行距为单倍行距,图的编号与图题文字之间空一个汉字符宽度。

图中标注的文字宜采用9~10.5pt,以能够清晰阅读为标准,且全文保持一致。专用名字代号、单位可采用外文表示,坐标轴题名、词组、描述性的词语均须采用中文。考虑到图的复制效果和成本等因素,图中不同序列的点、线、条块等宜使用不同形状、线型、填充图案等加以区分,尽量避免使用颜色区分。

如果一个图由两个或两个以上分图组成时,各分图分别以(a)、(b)、(c)……作为图序,并须有分图题。如果分图编号嵌在图中,字号可略大于图中标注文字的字号以示区别,并保持全文一致;如果置于图片下方,字号宜与图的编号和图题字号保持一致。

图宜紧置于首次引用该图的文字之后。图应尽可能显示在同一页(屏)。如图太宽,可逆时针方向旋转 90°放置。图页面积太大时,可分别配置在两页上,次页上应注明“(续)”,并注明图题(可省略),例如“图 2.1(续) 发展中国家经济增长速度的比较(1960-2000)”。

如需英文图名,应中英文对照,英文图的编号与图名另起一行放在中文下方。英文编号和内容应和中文一致,如“Fig 2.1 Comparison of economic growth rates indeveloping countries (1960-2000)”

\noindent 2. 表

表应具有“自明性”。表中参数应标明量和单位的符号。表头中应标明量和单位表示符号,表中的数字后面不再加单位符号。建议采用三线表,表的上、下边线为单直线,线粗为1.5磅;第三条线为单直线,线粗为1磅。

表单元格中的文字一般应居中书写(上下居中,左右居中),不宜左右居中书写的,可采取两端对齐的方式书写。表单元格中的文字采用11pt字书写,汉字用宋体,英文和数字用Times New Roman体,单倍行距,段前空3磅,段后空3磅。

表应有编号与表题,例如:“表3.1 第四次全国经济普查数据(北京)”。 表3.1是编号,代表“第3章第1个表”,以此类推,“第四次全国经济普查数据(北京)”是表题。表的编号与表题置于表上方,采用11pt字居中书写,汉字用宋体,英文和数字用Times New Roman 体,段前空12磅,段后空6磅,行距为单倍行距,表序与表题文字之间空一个汉字符宽度。

如某个表需要转页接排,在随后的各页上应重复表的编号,编号后跟表题(可省略)和“(续)”,置于表上方,例如“表3.1(续) 第四次全国经济普查数据(北京)”,续表均应重复表头和关于单位的陈述。

若在表下方注明资料来源,则此部分用五号字,汉字用宋体,英文用Times New Roman 体,单倍行距,段前空6磅,段后空12磅。需要续表时,资料来源注明在续表之下。

如需英文表名,应中英文对照,英文表的编号与表名另起一行放在中文下方。英文编号和内容应和中文一致,如“Table 3.1 Data from the Fourth National Economic Census (Beijing)”。

\noindent 3. 表达式

表达式主要是指数字表达式,例如数学表达式,也包括文字表达式。

表达式应另起一行,采用与正文相同的字号居中书写,或另起一段空两个汉字符书写,一旦采用了上述两种格式中的一种,全文都要使用同一种格式。表达式应有编号,编号应加括号置于表达式右边行末,编号与表达式之间不加任何连线。

较长的表达式必须转行时,应在“=”或者“+”“-”“×”“/”等运算符或者“]”“\}”等括号之后回行。上下行尽可能在“=”处对齐。

表达式采用Cambria Math或Times New Roman体,采用12pt字书写,行距为单倍行距,段前空6磅,段后空6磅。

当表达式不是独立成行书写时,应尽量将其高度降低为一行,例如,将分数线书写成“/”,将根号改为负指数,例如2\textsuperscript{-1/2}。

\noindent\textbf{(十)参考文献}

参考文献是文中引用的有具体文字来源的文献集合,列出作者直接阅读过、在正文中被引用过、正式或非正式发表的刊物、文献及资料。参考文献的写法应遵循国家标准《信息与文献参考文献著录规则》(GB/T 7714—2015);符合特定学科的通用范式,可使用APA或《清华大学学报(哲学社会科学版)》格式,且应全文统一,不能混用。参考文献一律放在论文结论后,不得放在各章之后。在论文正文中引用了参考文献的部位,须用上标标注[参考文献序号]。

“参考文献”四个字的格式与章标题的格式相同。参考文献表的正文部分用五号字,中文用宋体,英文和数字用Times New Roman体,行距采用固定值16磅,段前空3磅,段后空0磅。采用悬挂格式,悬挂缩进2个汉字符或1厘米。

每一条文献的内容要尽量写在同一页内。遇有被迫分页的情况,可通过“留白”或微调本页行距的方式尽量将同一条文献内容放在一页。

关于参考文献国家标准《信息与文献参考文献著录规则》(GB/T 7714—2015)的著录格式以及在正文中的标注方法详见《清华大学综合论文训练写作规范(试行)》附件的《参考文献著录规则及注意事项》。

\noindent\textbf{(十一)附录}

附录的格式与正文相同,并依顺序用大写字母 A,B,C,……编序号,如附录A,附录B,附录C,……。只有一个附录时也要编序号,即附录A。每个附录应有标题。附录序号与附录标题之间空一个汉字符。例如:“附录A 外文资料的调研阅读报告”。

附录中的图、表、数学表达式、参考文献等另行编序号,与正文分开,一律用阿拉伯数字编码,但在数码前冠以附录的序号,例如“图A.1”,“表B.2”,“式(C-3)”等。

附录内容分为以下两部分:

1、附录A

附录A为外文资料的调研阅读报告或书面翻译。调研阅读报告需附参考文献;书面翻译需注明外文资料原文的索引。标题为“外文资料的调研阅读报告”或“外文资料的书面翻译”。

调研阅读报告的参考文献(或书面翻译对应的外文资料的原文索引)格式与正文参考文献格式相同。

2、其他附录

其他附录是与论文内容密切相关、但编入正文又影响整篇论文编排的条理和逻辑性的资料,例如某些重要的数据表格、计算程序、统计表等,是论文主体的补充内容,可根据需要设置。其他附录序号为附录B,附录C,……。

\noindent\textbf{(十二)致谢}

致谢包括内容如:对国家科学基金、资助研究工作的奖学金基金、合同单位、资助或支持的企业、组织或个人,对协助完成研究工作和提供便利条件的组织或个人,对在研究工作中提出建议和提供帮助的人,对给予转载和引用权的资料、图片、文献、研究思想和设想的所有者,对其他应感谢的组织和个人。

致谢单设一页。标题为“致谢”。内容部分采用宋体小四号字,行距用固定值20 磅,段前后0磅。

\noindent\textbf{(十三)声明}

关于论文内容没有侵占他人著作权的声明,放在致谢页后,单独一页。标题为“声明”。内容为“本人郑重声明:所呈交的综合论文训练论文,是本人在导师指导下,独立进行研究工作所取得的成果。尽我所知,除文中已经注明引用的内容外,本论文的研究成果不包含任何他人享有著作权的内容。对本论文所涉及的研究工作做出贡献的其他个人和集体,均已在文中以明确方式标明。” 确认无误后,慎重签名。

此部分内容可以直接下载《清华大学综合论文训练写作规范(试行)》附件的WORD文档,相应地复制到论文中即可,在提交论文时作者必须签署姓名。

\noindent\textbf{(十四)在学期间参加课题的研究成果}

指在本科阶段课题研究中获得的成果,如申请的专利或已正式发表和已有正式录用函的论文等。标题为“在学期间参加课题的研究成果”, 使用黑体三号字,居中书写,单倍行距,段前空24磅,段后空18磅。内容部分用宋体小四号字,行距采用固定值20磅,段前后0磅。

各种类型学术成果的书写格式与正文相同,书写要求如下。

1. 学术论文

参照参考文献书写,尚未刊载但已经接到正式录用函的学术论文加括号注明已被××××期刊录用。

2. 专著/译著

参照参考文献书写,尚未出版但已被出版社决定出版的专著/译著加括号注明出版社名称和预计出版时间。

3. 专利

参照参考文献书写,处于申请阶段的专利在专利号位置填写专利申请号,并加括号注明是专利申请号。

4. 作品

大致按以下方式书写:作者. 作品名称. 创作时间. 材料形式. 作品尺寸. 作品地点. 参展信息. 是否获奖等信息。

5. 研究报告

公开的研究报告参照参考文献书写。

6. 其他

按适当合理的方式书写。

\noindent\textbf{(十五)综合论文训练记录表}

完整和翔实记录综合论文训练开题、中期检查、论文评阅、论文答辩各环节情况。装订于论文的最后。

\noindent\textbf{(十六)页面设置}

论文页面设置如下:
\begin{enumerate}
  \item 封面:纸张规格、尺寸,A4(21厘米×29.7厘米);页边距,上3.8厘米,下3.2厘米,左3厘米,右3厘米;装订线,0.2厘米,位置左。
  \item 除封面外,其他页面:纸张规格、尺寸,A4(21厘米×29.7厘米);页边距,上3厘米,下3厘米,左3厘米,右3厘米;装订线,0厘米。
\end{enumerate}
注:论文除“封面”、“关于论文使用授权的说明”采用单面印刷之外,从摘要开始(包括摘要)后面的部分均采用A4幅面白色70克以上80 克以下(彩色插图页除外)纸张双面印刷,正文从另页右页开始。

\noindent\textbf{(十七)页眉和页码}

页眉:无

页码:位于页面底端,居中书写。在第1章(引言或绪论)之前的部分,从前往后用大写罗马数字编排;从第 1 章(引言或绪论)开始,用阿拉伯数字连续编排。综合论文训练记录表无页码。

\noindent\textbf{(十八)书脊的书写要求}

用仿宋字书写,字体大小根据论文的薄厚而定。书脊上方写论文题目,下方写作者姓名,距上下页边均为3cm。

\noindent\textbf{(十九)其它说明}

论文的某些部分内容若为空,如:主要符号表、在学期间参加课题的研究成果、附录B等,则该部分不要作为空白页装订在论文里。论文封皮统一要求使用120克蓝色纸。

\section{模板及相关说明}
本科生综合论文训练论文模板以附件形式单独存成文档,供同学们下载参考。如果模板中存在与本规范中的规定不符之处,以本规范中的文字叙述为准。

\noindent 01 综合论文训练论文模板

\noindent 02 关于论文使用授权的说明

\noindent 03 声明

\noindent 04 在学期间参加课题的研究成果

\noindent 05 综合论文训练记录表

\noindent 06 参考文献著录规则及注意事项

\chapter{关于模板的一些说明}
\section[tplstruct]{全文框架}
在本模板类(documentclass\{Diploma\})的支持下,综合论文训练全文框架应遵循以下基本规范:
\begin{verbatim}
\documentclass{Diploma}
\usepackage{...}
\SetDepartment{系名}
\SetMajor{专业名}
\SetTitle{论文题名}
\SetAuthor{作者名}
\SetInstructor{教师名}{教师职称}
\begin{document}
\ChsAbstract
%中文摘要内容
\ChsKeywords{中文关键词}
\EngAbstract
%英文摘要内容
\EngKeywords{英文关键词}
\ListOfContents%目录及图表清单
\begin{Abbreviations}
%符号和缩略语列表
\end{Abbreviations}
\StartMainText%正文开始
%正文内容
\bibliography{references}%参考文献
\StartAppendix%附录开始
%附录内容
\Acknowledgments%致谢开始
%致谢内容
\Statement%声明
\Achievements%在学期间参加课题的研究成果
\end{document}
\end{verbatim}

\section[bibref]{参考文献库说明}
参考文献库文件名为“*.bib”,需要在文档中使用“\char92 bibliography”显式地进行引用。“*.bib”文献库文件的基本格式为:
\begin{verbatim}
@entry{code,
   info = {content},
}
\end{verbatim}
其中各项参数的说明如下:
\begin{itemize}
  \item entry,文献类型。常用的是这四种类型:“article”期刊论文、“inproceedings”会议论文、“book”书籍、“thesis”学位论文。不常用的类型还有这些:“report”报告、“standard”标准、“incollection”汇编中的论文、“proceedings”会议论文集、“collection”论文汇编集、“inbook”书籍中的章节、“periodical”连续出版物、“news”新闻。
  \item code,文献代码,具有唯一性的代码,以便“\char92 cite”命令进行引用。
  \item info,文献信息名。
  \item content,文献信息的实际内容。
\end{itemize}
下面将详细列出各种不同类型文献所包含的数据信息及其说明。

\textbf{“article”期刊论文}可包含以下文献信息:
\begin{itemize}
  \item author,作者名。必填项。多个作者名之间用“and”隔开,如“author = \{作者甲 and 作者乙\},”。
  \item title,论文题名。必填项。
  \item journal,期刊名。必填项。
  \item year,出版年。必填项。原则上,期刊的“年卷期”三项信息均应填写。如确有缺失,至少要填写一项。
  \item volume,卷号。必填项。
  \item number,期号。必填项。
  \item pages,页码范围。必填项。给出论文在期刊中的页码范围。
  \item title.aux,其他期刊名。选填项。期刊的“别称”,一般不填。
  \item date,引用日期。选填项。一般不填。
  \item url,电子资源网址。选填项。
  \item doi,唯一的数字对象标识符。选填项。
  \item language,文献语言。选填项。设成“cn”为中文,否则均默认为英文。
\end{itemize}

\textbf{“inproceedings”会议论文}是指会议论文集中的论文,含以下信息:
\begin{itemize}
  \item author,论文作者名。必填项。多个作者名之间用英文“and”隔开。
  \item author.aux,论文其他作者。选填项。
  \item title,论文名。必填项。
  \item pages,页码范围。必填项。给出论文的具体页码范围。
  \item editor,会议论文集编撰者。选填项。多个编者之间用逗号隔开。
  \item booktitle,会议论文集名称。必填项。
  \item title.aux,会议论文集别名。选填项。
  \item edition,会议论文集版本信息。选填项。会议论文集一般没有版本信息。
  \item address,会议举办地。必填项。
  \item publisher/organization,出版者或会议组织机构。必填项。
  \item year,会议举办年。选填项。
  \item date,引用日期。选填项。一般不填。
  \item url,电子资源网址。选填项。
  \item doi,唯一的数字对象标识符。选填项。
  \item language,文献语言。选填项。
\end{itemize}

\textbf{“book”书籍}可包含以下文献信息:
\begin{itemize}
  \item author,作者名。必填项。作者名之间用“and”隔开。
  \item title,书名。必填项。
  \item address,出版地。必填项。电子资源可省略出版地信息。
  \item publisher,出版者。必填项。电子资源可省略出版者信息。
  \item year,出版年。选填项。
  \item author.aux,其他作者。选填项。此信息常见于译著,可以设置“author”为原文献作者,设置“author.aux”为译者。
  \item title.aux,其他书名。选填项。书籍“别称”,一般不填。
  \item edition,版本信息。选填项。首个版本通常不填写版本信息。
  \item pages,页码范围。选填项。
  \item date,引用日期。选填项。一般不填。
  \item url,电子资源网址。选填项。电子资源可不填出版地、出版者信息。
  \item doi,唯一的数字对象标识符。选填项。
  \item language,文献语言。选填项。
\end{itemize}

\textbf{“thesis”学位论文}可包含以下文献信息:
\begin{itemize}
  \item author,作者名。必填项。
  \item title,学位论文题名。必填项。
  \item address,取得学位论文的城市。必填项。
  \item publisher/organization,取得学位论文的大学或其他学术单位。必填项。
  \item year,取学位的年份。必填项。
  \item title.aux,学位论文的其他题名信息。选填项。
  \item pages,页码范围。选填项。
  \item date,引用日期。选填项。一般不填。
  \item url,电子资源网址。选填项。
  \item doi,唯一的数字对象标识符。选填项。
  \item language,文献语言。选填项。
\end{itemize}
因为国标文件并不明确区分学位论文的类型,所以这里将“mastersthesis”和“phdthesis”设置为“thesis”的别名,三者功能上完全相同。

\textbf{“report”技术报告}是指独立成册的专项报告,可包含以下文献信息:
\begin{itemize}
  \item author,作者名。必填项。
  \item title,报告名。必填项。
  \item edition,版本信息。选填项。
  \item address,出版地。必填项。
  \item publisher,出版者。必填项。
  \item author.aux,其他作者。选填项。
  \item title.aux,其他报告名。选填项。
  \item year,出版年。选填项。
  \item pages,页码范围。选填项。
  \item date,引用日期。选填项。一般不填。
  \item url,电子资源网址。选填项。
  \item doi,唯一的数字对象标识符。选填项。
  \item language,文献语言。选填项。
\end{itemize}

\textbf{“standard”国家/国际标准}是指标准化规范,可包含以下文献信息:
\begin{itemize}
  \item author,作者名。必填项。
  \item title,标准名。必填项。
  \item edition,版本信息。选填项。
  \item address,出版地。必填项。
  \item publisher,出版者。必填项。
  \item author.aux,其他作者。选填项。
  \item title.aux,标准别名。选填项。
  \item year,出版年。选填项。
  \item pages,页码范围。选填项。
  \item date,引用日期。选填项。一般不填。
  \item url,电子资源网址。选填项。
  \item doi,唯一的数字对象标识符。选填项。
  \item language,文献语言。选填项。
\end{itemize}

\textbf{“incollection”汇编论文}是指汇编中的独立论文,可包含以下文献信息:
\begin{itemize}
  \item author,论文作者名。必填项。
  \item author.aux,论文其他作者。选填项。
  \item title,论文名。必填项。
  \item pages,页码范围。必填项。给出论文的具体页码范围。
  \item editor,论文汇编编撰者。选填项。
  \item booktitle,论文汇编名称。选填项。
  \item title.aux,论文汇编别名。选填项。
  \item edition,论文汇编版本信息。选填项。
  \item address,出版地。必填项。
  \item publisher,出版者。必填项。
  \item year,出版年。选填项。
  \item date,引用日期。选填项。一般不填。
  \item url,电子资源网址。选填项。
  \item doi,唯一的数字对象标识符。选填项。
  \item language,文献语言。选填项。
\end{itemize}

\textbf{“proceedings”会议论文集}一般指举办学术会议时产生的会议论文集合,有些会议论文集是公开出版的,有些只对参会者内部分发。可包含以下文献信息:
\begin{itemize}
  \item author,论文集编者。必填项。
  \item author.aux,论文集其他编者。选填项。
  \item title,论文集名称。必填项。
  \item title.aux,会议论文集别名。选填项。
  \item edition,会议论文集版本信息。选填项。会议论文集一般没有版本信息。
  \item address,会议举办地。必填项。
  \item publisher/organization,出版者或会议组织机构。必填项。
  \item year,会议举办年。选填项。
  \item pages,页码范围。选填项。
  \item date,引用日期。选填项。一般不填。
  \item url,电子资源网址。选填项。
  \item doi,唯一的数字对象标识符。选填项。
  \item language,文献语言。选填项。
\end{itemize}

\textbf{“collection”汇编论文集}一般指独立成册的专题论文集,包含以下信息:
\begin{itemize}
  \item author,论文集编者。必填项。
  \item author.aux,论文集其他编者。选填项。
  \item title,论文集名称。必填项。
  \item title.aux,论文集别名。选填项。
  \item edition,论文集版本信息。选填项。
  \item address,出版地。必填项。
  \item publisher,出版者。必填项。
  \item year,出版年。选填项。
  \item pages,页码范围。选填项。
  \item date,引用日期。选填项。一般不填。
  \item url,电子资源网址。选填项。
  \item doi,唯一的数字对象标识符。选填项。
  \item language,文献语言。选填项。
\end{itemize}

\textbf{“inbook”书籍内容}一般指书籍中的某个具体的章节,可包含以下文献信息:
\begin{itemize}
  \item author,章节作者名。必填项。有些书籍每个章节都有单独的作者。
  \item author.aux,章节其他作者。选填项。
  \item title,章节名。必填项。
  \item pages,页码范围。必填项。给出所引用的章节的具体页码范围。
  \item editor,书籍的总编撰者。选填项。若总编与章节作者相同,则不必填此项。
  \item booktitle,书籍名。必填项。
  \item title.aux,书籍别名。选填项。
  \item edition,书籍版本信息。选填项。
  \item address,出版地。必填项。
  \item publisher,出版者。必填项。
  \item year,出版年。选填项。
  \item date,引用日期。选填项。一般不填。
  \item url,电子资源网址。选填项。
  \item doi,唯一的数字对象标识符。选填项。
  \item language,文献语言。选填项。
\end{itemize}

\textbf{“periodical”连续出版物}一般指期刊本身,引用期刊本身是不常见的,一般情况下只引用刊物的某一期中的某一篇文章。可包含以下文献信息:
\begin{itemize}
  \item author,期刊编者名。必填项。
  \item title,期刊名。必填项。
  \item title.aux,期刊别名。选填项。
  \item year,起始出版年。选填项。
  \item volume,起始卷号。必填项。
  \item number,起始期号。必填项。
  \item year,终出版年。选填项。
  \item vol.end,终卷号。必填项。
  \item num.end,终期号。必填项。
  \item address,出版地。必填项。
  \item publisher,出版者。必填项。
  \item date,引用日期。选填项。一般不填。
  \item url,电子资源网址。选填项。
  \item doi,唯一的数字对象标识符。选填项。
  \item language,文献语言。选填项。
\end{itemize}

\textbf{“news”新闻}一般指报刊上的新闻信息,工科论文一般不引用新闻。
\begin{itemize}
  \item author,新闻作者名。必填项。
  \item title,新闻标题。必填项。
  \item newspaper,报刊名。必填项。
  \item title.aux,报刊别名。选填项。
  \item year,出版年。选填项。
  \item number,期号。必填项。
  \item pages,页码范围。必填项。给出所引用的新闻的具体页码范围。
  \item date,引用日期。选填项。一般不填。
  \item url,电子资源网址。选填项。
  \item doi,唯一的数字对象标识符。选填项。
  \item language,文献语言。选填项。
\end{itemize}

\section{不够优雅的dirty trick}
熟练的程序员常常会听到“dirty trick”这个英文词组,直译成中文是“肮脏的把戏”。实际上,这个词组主要是指那些临时性的、非标准的或取巧的解决方案。在本文提供的模板中,也有一些这种临时性的、未经过全面兼容性检验的“dirty trick”。本节将给出一些说明,以备不时之需。

\noindent\textbf{关于字体}

本模板中会用到5种字体:
\begin{itemize}
  \item 字体名Times New Roman,字体文件times.ttf,以及加粗的字体timesbd.ttf。
  \item 字体名Arial,字体文件arial.ttf。
  \item 字体名SimSun,字体文件simsun.ttc。这是中文宋体字,主要用于正文内容。
  \item 字体名SimHei,字体文件simhei.ttf。这是中文黑体字,主要用于标题。
  \item 字体名FangSong,字体文件simfang.ttf。这是中文仿宋体,只用于封面的作者信息等内容。
  \item 字体名DengXian,字体文件deng.ttf。这是中文等线字体,姑且认为这种字体与\LaTeX 中的英文mono字体对应。等线字体主要用于展示源代码,“规范”中对此部分没有硬性规定。
\end{itemize}
以上字体是Windows系统中自带的字体。对于Mac和基于Linux的操作系统,需要手动安装以上字体。或者修改“Diploma.cls”中的相关定义,以使用其他系统中已有的字体。但按“规范”要求,原则上应使用以上字体。

\noindent\textbf{关于数学公式}

在科技界,\LaTeX 是以其强大的数学公式编辑能力而闻名的。为了进一步扩充公式编辑的便利性和规范性,美国数学学会(American Mathematical Society,简称AMS)开发并维护了“amsmath”这个编辑复杂公式时一定会用到的宏包。本模板没有默认引用此宏包,需要使用时可在“\char92 documentclass”命令之后使用“\char92 usepackage\{amsmath\}”命令来引入该宏包,以便使用其提供的丰富功能。

\noindent\textbf{关于功能宏包}

由于\TeX 的开放性(实际情况是\TeX 核心功能太少不够用),全世界有大量的热心人参与开发各式各样的\LaTeX 宏包。本模板默认引入了以下4个宏包:
\begin{itemize}
  \item xeCJK:支持东亚字符(中日韩)。实际上主要是解决了中文字体的设置、汉字间距控制等问题。
  \item graphics:支持在文档中插入图片。此宏包支持大部分的图片格式,jpg、png、tif之类的,应该能够应对绝大部分情况。
  \item longtable:支持长表格(即跨页的表格)。本模板的“符号和缩略语”部分使用了“长表格”功能。
  \item hyperref:为PDF文档添加“链接”功能。例如,点击目录中的章节标题就能直接跳转到对应章节,这是很方便的。不光是目录,所有使用到“引用”的地方,此宏包都会自动加上“链接”功能。
\end{itemize}
实际上,以上4个宏包还会关联引入大量宏包,编译时有相关信息提示。

大量的\LaTeX 宏包固然能够解决很多实际问题,但有时宏包之间有产生“冲突”,出现兼容性问题。毕竟来自世界各地的开发者脾气秉性各不相同,又缺乏强大的组织管理,兼容性问题是不可避免的。相对而言,Word软件的兼容性问题就少得多。考虑到这个因素,在开发本模板时,我们并没有大量的引入宏包,而是尽量只使用标准的、原始的\LaTeX 命令来解决问题。

实际使用时不免需要额外的宏包来解决复杂的排版需求。在“\char92 documentclass”命令之后使用“\char92 usepackage”命令即可引入宏包并实现相应的功能。需要特别说明的是,由于本模板对一些基本命令进行了深度定制,可能会跟某些宏包产生不可控的兼容性问题。欢迎反馈这些兼容性问题,但我们不保证能够解决它。

\noindent\textbf{尽量不自行使用figure或table环境}

有丰富的\LaTeX 使用经验的同学,一定会经常用到figure和table环境。在本模板中,没有给出figure环境的定义,所以如果之前使用了figure环境来插入图片,则需要手动转换为本模板中的“\char92 InsertFigure”、“\char92 MultiFigure”、“\char92 SubFigure”等命令。转换并不困难,只是需要花一些时间。

本模板给出了table环境的定义,但与标准的\LaTeX\ table完全不同。如果之前使用了标准的table环境来插入表格,这里仍然需要进行转换。表格的转换相对会比较繁琐,实在遇到搞不定的问题请及时联系我们并尽量解决。

\noindent\textbf{控制不住的“浮动体”}

在\LaTeX 中,“浮动体”是一个重要的概念,主要指“游离”于文字之外的“图”和“表”。在本模板中,“\char92 InsertFigure”、“\char92 begin\{multifigures\}”、“\char92 begin\{subfigures\}”、“\char92 begin\{table\}”等命令就会产生浮动体。

通常,\LaTeX 系统会按照一般的排版逻辑,根据尽量使“页面充实”的原则,把浮动体排到合适的位置。这个“合适的位置”可能是“伪代码”中浮动体出现的位置,也可能是页面的顶端,也可能是页面的底端,还可能是隔了一页甚至好几页后才出现。总之,给人的感觉是,浮动体出现的位置不可控。其实,这是因为在\LaTeX 允许浮动体脱离上下文,放置在算法认为合适的位置。在技术类论文中,有一种常见的情况:需要根据上下文在某处插入较多的图表。在此算法下,这些图表很有可能会被\LaTeX 放置到远离上下文的页面,从而对读者造成一定的困扰。为了避免出现此类情况,可以使用“\char92 clearpage”命令,强制\LaTeX 系统分页,以不好看的“留白”换取浮动体尽量出现在上下文的近处。如果由此产生的“留白”过大,那么就需要手工精调文字段落来“消除”留白。总之,自动排版算法并不能很好的解决此类问题,作者需要在论文终版时进行手动精调。

本模板按照通常的经验,将浮动体的位置控制选项设置为“!htb”。其中“!”表示排版浮动体时尽量不受算法限制,即更倾向于满足使用者的需求;“htp”表示优先把浮动体排在对应的上下文之中,实在排不下再考虑排到页面顶端或底端。

\noindent\textbf{不守套路的空白}

在\ref{txt:usage}中曾提及,中文\LaTeX “伪代码”里的空格会被自动忽略掉,不对排版产生影响;但中英文之间会自动插入合适的间距。这一机制能够保障大多数时候的排版效果,但偶尔也会出现失效的场景。例如“\char92 LaTeXe”这个命令,若其后紧跟的是中文字符,则按排版规则应自动插入一段“中英文间距”,但实际的排版效果却不是这样。以下代码
\begin{verbatim}
\LaTeXe 汉
\end{verbatim}
的排版效果是“\LaTeXe 汉”,可见其中的字符$\varepsilon$与“汉”字之间是缺少合适的空白间距的。此时就需要显式地加入间距,以下代码
\begin{verbatim}
\LaTeXe\ 汉
\end{verbatim}
的排版效果“\LaTeXe\ 汉”才是正确的。

在另一些情形中,可能会多出空白间距。例如,当我们在上下文中间插入浮动体时,原则上浮动体的前后文字应当是“紧挨”着没有间距的,但却因伪代码换行的问题而产生了多余的间距。例如以下代码
\begin{verbatim}
前
\InsertFigure{}
后
\end{verbatim}
会在“前”、“后”两字之间插入多余的空白间距,这是因为花括号“\}”后的换行符会产生一个“空白”。此时应将代码修正为
\begin{verbatim}
前
\InsertFigure{}%
后
\end{verbatim}
即在“\}”后添加注释符“\%”以便强制消除之后可能会产生的多余空白。

\noindent\textbf{不那么如意的垂直间距}

按“规范”要求,正文段落的行距为固定值20磅。在\TeX 中,行距控制的是“基线间距”,即两行的基线间距离保持20pt。这里就产生了一个问题,对于每一页的第一行,它的基线就是本页面的首个基线,这首个基线应该放什么位置呢?\TeX 系统的处理方式是把基线放到最“靠上”的位置,即把文字顶到最上方放置,并“吃”掉所有的垂直间距。所以,对于每一页的首行文字,会出现以下现象:
\begin{itemize}
  \item 段前垂直间距失效。例如“规范”要求一级节标题段前空24磅,若节标题为首行,实际上不存在这个段前间距。当然,这样的处理方式是符合逻辑的。
  \item 当\TeX 把“浮动体”放置在页面最上方时,“首行文字”仍然是首行文字,段前的垂直空间会失效。此时,“浮动体”与文字的间距会略小,因为本来应该存在的段前间距被“吃”掉了。这就是当图表出现在页首时,图表与文字的间距看起来会略微小一些的原因。本模板已经通过各种方式把这一影响减到最小了。
\end{itemize}

另一方面,按“规范”要求,章标题段后空18磅,一级节标题段前空24磅。因此,当章标题后紧跟一级节标题时,两者间会空出42磅,这当然是比较大的间距。所以,在本模板中,在处理节标题的段前间距时,会先用“\char92 removelastskip”命令取消上一个垂直间距,以避免垂直间距的叠加。

当浮动体出现在页顶时,其与页顶的间距由“\char92 @fptop”控制,\LaTeX 将它默认定义为“0pt plus 1fil”,即:间距默认为0,但可以按需填充。因此,当有浮动体出现在页顶,\LaTeX 系统通常会把所有的可调整间距全部放到页顶与浮动体之间。换言之,此时浮动体与页顶的间距会稍大一些。同样的,当浮动体出现在页底时,间距由“\char92 @fpbot”控制。

\noindent\textbf{BibTeX的逆波兰表达式}

BibTeX的参考文献格式定义文件“*.bst”实际也是一种“程序”,这种“程序”使用了一种比较少见的表达方式,即“逆波兰表达式”。

以加法为例,按人类的思维结构,一个正常的加法表达式应该是这样的“a+b”,即两个“操作数”分别出现在“操作符”的两侧,这就叫“波兰表达式”(Polish Notation)。但是,这种人类容易理解的表达式结构,对于“计算机编译系统”来说,是不方便理解的。对“计算机编译系统”来说,更容易“理解”的形式是:把“操作数”集中放前面,“操作符”放到所有“操作数”之后。这就叫“逆波兰表达式”(Reverse Polish Notation)。例如,波兰表达式“a+b”写成逆波兰表达式的形式应该是“a b +”。

理解了“逆波兰表达式”的特点,就比较容易看懂“bst”文件了。

\Acknowledgments%致谢
衷心感谢指导教师××教授对本人的精心指导。他的言传身教将使我终生受益。

感谢×××××实验室××教授,以及×××××全体老师和同窗们的热情帮助和支持!

……

本课题承蒙×××××基金资助,特此致谢。

以上是“规范”推荐的致谢内容形式。本模板特别感谢马正华、闫济洲、许瀵译、郑皓文同学(排名不分先后)参与调试并提出修订建议。

\Statement%声明

\Achievements%在学期间参加课题的研究成果
\begin{MyPapers}
  \item 	ZHOU R, HU C, OU T, et al. Intelligent GRU-RIC Position-Loop Feedforward Compensation Control Method with Application to an Ul-traprecision Motion Stage[J], IEEE Transactions on Industrial Infor-matics, 2024, 20(4): 5609-5621.
  \item 	杨轶, 张宁欣, 任天令, 等. 硅基铁电微声学器件中薄膜残余应力的研究[J]. 中国机械工程, 2005, 16(14):1289-1291.
  \item YANG Y, REN T L, ZHU Y P, et al. PMUTs for handwriting recogni-tion. In press[J]. (已被Integrated Ferroelectrics录用)
\end{MyPapers}

\vspace{20pt}
\begin{MyPatents}
  \item 	胡楚雄, 付宏, 朱煜, 等. 一种磁悬浮平面电机: ZL202011322520.6[P]. 2022-04-01.
  \item 	REN T L, YANG Y, ZHU Y P, et al. Piezoelectric micro acoustic sensor based on ferroelectric materials: No.11/215, 102[P]. (美国发明专利申请号.)
\end{MyPatents}

\end{document}
